% Kapitola Teorie
\chapter{Teorie TODO Rename}

\begin{chapterabstract}
	V této kapitole se budeme zabývat teoretickými východisky práce a vymezením používaných pojmů.
	 Také si ukážeme existující řešení problémů podobných našemu.
	 TODO Doplnit abstract Teorie
\end{chapterabstract}

\section{Analýza sentimentu}
Na poli výzkumu dolování dat z textu můžeme narazit na dva pojmy, které jsou zaměnitelné. Jedná se o \textit{Analýzu sentimentu (SA)} a \textit{Opinion mining (OM)}. \cite{Medhat} Někteří autoři je ale rozlišují. Například Tsytsarau a Palpanas uvádějí \cite{survey}, že OM extahuje a analyzuje názor lidí vázající se k nějakému objektu či entitě, kdežto SA pouze identifikuje sentiment vyjádřený v textu a ten pak analyzuje. \marginpar{\todo{pozn. redakce: Víc mi teda sedí to vyjádření že SA a OM jsou jiné, ale uvidim}}
Podle Medhata a spol. můžeme SA rozdělit na 3 klasifikační úrovně \cite{Medhat}:
\begin{description}
	\item[Document-level] Základní jednotkou informace je jeden dokument (týkající se jednoho tématu)
	
	\item[Sentence-level] Určuje sentiment na základě jedné věty.
	
	\item[Aspect-based] Určuje sentiment s ohledem na entity a jejich aspekty, kde autoři názoru mohou poskytovat různé názory na různé aspekty jedné entity (např. \uv{V této firmě se necítím dobře, ale platové podmínky jsou nejlepší na trhu}).
\end{description}

Feldman k těmto úrovním přidává ještě další 2 \cite{Feldman}:
\begin{description}
	\item[Komparativní] Základní jednotkou informace je jeden dokument (týkající se jednoho tématu)
	
	\item[Lexicon acquisition] Určuje sentiment na základě jedné věty. Není nijak zvláště rozdílná od Dokument-level analýzy, protože věty jsou v principu jen krátké dokumenty. 
	
\end{description}

\subsection{Document-level}
\blind[2]
\subsection{Sentence-level}
\blind[2]
\subsection{Aspect-level}
Není nijak zvláště rozdílná od Dokument-level analýzy, protože věty jsou v principu jen krátké dokumenty. 
\blind[2]
\subsection{Komparativní analýza}
\blind[2]
\subsection{Lexicon acquisition}
\blind[2]
		

