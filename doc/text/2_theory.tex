% Kapitola Teorie
\chapter{Teoretická východiska práce}

\begin{chapterabstract}
	V této kapitole se budeme zabývat teoretickými východisky práce a vymezením používaných pojmů.
	 Také si ukážeme existující řešení problémů podobných našemu.
	 TODO Doplnit abstract Teorie
\end{chapterabstract}

\section{Analýza sentimentu}
Na poli výzkumu dolování dat z textu můžeme narazit na dva pojmy, které jsou zaměnitelné. Jedná se o \textit{Analýzu sentimentu (SA)} a \textit{Opinion mining (OM)}. \cite{Medhat} Někteří autoři je ale rozlišují. Například Tsytsarau a Palpanas uvádějí \cite{survey}, že OM extahuje a analyzuje názor lidí vázající se k nějakému objektu či entitě, kdežto SA pouze identifikuje sentiment vyjádřený v textu a ten pak analyzuje. \marginpar{\todo{pozn. redakce: Víc mi teda sedí to vyjádření že SA a OM jsou jiné, ale uvidim}}
Podle Medhata a spol. můžeme SA rozdělit na 3 klasifikační úrovně \cite{Medhat}:
\begin{description}
	\item[Document-level] Základní jednotkou informace je jeden dokument (týkající se jednoho tématu)

	\item[Sentence-level] Určuje sentiment na základě jedné věty. 
	
	\item[Aspect-based] Určuje sentiment s ohledem na entity a jejich aspekty, kde autoři názoru mohou poskytovat různé názory na různé aspekty jedné entity (např. \uv{V této firmě se necítím dobře, ale platové podmínky jsou nejlepší na trhu}).
\end{description}

Feldman k těmto úrovním přidává ještě další 2 \cite{Feldman}:
\begin{description}
	\item[Komparativní] Základní jednotkou informace je jeden dokument (týkající se jednoho tématu)
	
	\item[Lexicon acquisition] Určuje sentiment na základě jedné věty. Není nijak zvláště rozdílná od Dokument-level analýzy, protože věty jsou v principu jen krátké dokumenty. 
	
\end{description}

\subsection{Document-level}
Jedná se o nejjedodušší formu SA. Jak již bylo řečeno, základní jednotkou informace této metody je jeden dokument a předpokládá se, že zde autor vyjadřuje svůj názor na jeden objekt. Existují dva hlavní přístupy řešení a to: supervizované a nesupervizované učení. Supervizované učení v tomto případě předpokládá, že existuje konečná množina tříd, reprezentující hledaný sentiment.  V nejjedodušším případě si vystačíme s 2 třídama, konkrétně negativní a pozitivní sentiment. Obecně se tak jedná o nějakou diskrétní škálu hodnot. 
Nesupervizované učení využívá Semantic orientation (dále jen SO) \todo{SO = kapitola?} konkrétních slov nebo vět. 
\subsection{Sentence-level}
Pokud problém vyžaduje detailnější analýzu, než poskytuje Document-level, využijeme Sentence-level metodu. Prvotně musíme věty rozdělit na subjektivní a objektivní. Objektivní věty se pro další analýzu často nevyužívají, ale některé přístupy využívají oba typy vět. Po rozdělení můžeme opět využít buďto supervizované nebo nesupervizované učení, stejně tak jako u Dokument-level analýzy.
Narayanan a spol. ukázali, že je vhodné na různé věty používat různé metody \cite{Feldman24}. Konrétně věty různých typů: tázací, podmiňovací a sarkastické. 
\subsection{Aspect-level}
Předchozí 2 typy byly předpokládaly že se každý dokument nebo věta váže k jednomu objektu. V mnoha případech však lidé hovoří o více objektech s více aspekty (objektem může být např. mobilní telefon, u kterého se můžeme vyjádřit k jeho aspektům (rychlost, velikost, paměť, atd.) kladně i záporně). Kdybychom analyzovali text o jednom objektu s více aspekty těmito přístupy, došlo by ke ztrátě mnoha informací. Aspect-level analýza se proto zaměřuje v na jednoltlivé aspekty objektu a k nim přiřazuje zjištěný sentiment. 

Nejdříve tedy musíme identifikovat aspekty objektu. Většina komerčních produktů používá přístup, kde si nejdříve vyfiltrují všechny fráze s podstatnými jmény (dále je n NPs -noun
phrases), kterých je určitý počet. \cite{Feldman} Dále pak můžeme zredukovat šum v NPs \todo{cite Feldman30} pomocí PMI \todo{PMI}. Existují ale i další přístupy: Například phrase dependency parser, Conditional Random Field, atd. Tyto metody identifikují takzvané explicitní aspekty, ale v textu se mohou nacházet i implicitní aspekty (např. ve větě: \uv{Toto auto je pomalé} není nikde uveden aspekt rychlosti).\todo{implicit metody} \uv{Finální polarita každého aspektu je determninována váženým průměrem polarit všech výrazů sentimentu nepřímo vážené vzdálenopsti  mezi aspektem a výrazem sentimentu.} \cite*[překlad vlastní]{Feldman}

\subsection{Komparativní analýza}
V některých případech se lidé nevyjadřují k objektech příme, ale porovnávají 2 objekty mezi sebou (např. \uv{Windows je lepší operační systém, než jakákoliv distribuce Linuxu}). Zde tedy musíme vyextahovat objekty z těchto vět. Autorům Jindal a Jiu se podařilo najít relativně malou množinu slov, která tyto věty dokáže identifikovat. \cite{comparative} Jedná se o anglické slova, například o: \cite{comparative} 
\begin{description}
	\item[Srovnávací přídavná jména a příslovce] \uv{more} - více, \uv{less} - méně a slova končící koncovkou \uv{-er}

	\item[Přídavná jména a příslovce v superlativu]  \uv{most} - nejvíce, \uv{least} - nejméně a slova končící koncovkou \uv{-est}
	
	\item[Jiné fráze] \uv{than} - než, \uv{outperform} - překonat, \uv{up against} - proti, $\dots$
\end{description}
\uv{Tyto klíčová slova a fráze (celkem 83) dokáží zachytit komparativní věty se senzitivitou $94\%$}. \cite[překlad vlastní]{comparative}


\subsection{Lexicon acquisition}
\blind[2]
		

